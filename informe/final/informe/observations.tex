 %%%%%%%% OBSERVACIONES %%%%%%%%%%%%%%%
 
 
Se afrontaron diversas inconvenientes y problemas que fueron surgiendo a lo largo de la implementación del amplificador. Entre ellas, hubieron fallas en soldaduras, que concluyeron con el cambio de los transistores de potencia en más de una ocasión. \\


Por otro lado, se debieron colocar capacitores de $220\si[per-mode=symbol]{\pico\farad}$ entre base y colector de los \textit{drivers} de los transistores de potencia para eliminar oscilaciones de alta frecuencia, estas oscilaciones no eran debidas a falta de compensación lineal, ya que las mismas solo se producían en uno de los ciclos de las señales y su intensidad y frecuencia eran distintas, dependiendo de si se trataba del semi-ciclo positivo o el negativo, de esto se concluyó que se trataba de inestabilidades alinéales y su solución con un capacitor entre base y colector de los drivers de los Darlington de salida pasa por disminuir la ganancia de los mismos a medida que la frecuencia aumenta, eliminando la posibilidad de realimentación positiva, su valor se determinó en forma empírica, y tratando de usar los valores menores posibles, ya que las simulaciones mostraban que el THD aumentaba en proporción a los valores de estos capacitores.\\


Luego, se debió fijar otro disipador al ya colocado, ya que la resistencia térmica necesaria para este trabajo no correspondía con la adquirida en el local.